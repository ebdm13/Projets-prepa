% My LaTeX template

%----------------------------------------------------------------------------------------
%    PACKAGES AND OTHER DOCUMENT CONFIGURATIONS
%----------------------------------------------------------------------------------------

\documentclass[11pt]{article} % Font size
\usepackage{amsmath, amsfonts, amssymb, amsthm}
\usepackage[french]{babel}
\usepackage{blindtext}
\usepackage{zref-totpages}
\usepackage{siunitx}
\usepackage{array}
\usepackage{graphicx}
\usepackage{subcaption}
\usepackage{float}
\usepackage{hyperref}
\usepackage{listings}
\usepackage{xcolor}
\definecolor{codegreen}{rgb}{0,0.6,0}
\definecolor{codegray}{rgb}{0.5,0.5,0.5}
\definecolor{codepurple}{rgb}{0.58,0,0.82}
\definecolor{backcolour}{rgb}{0.95,0.95,0.92}
\lstdefinestyle{mystyle}{
    backgroundcolor=\color{backcolour},   
    commentstyle=\color{codegreen},
    keywordstyle=\color{magenta},
    numberstyle=\tiny\color{codegray},
    stringstyle=\color{codepurple},
    basicstyle=\ttfamily\footnotesize,
    breakatwhitespace=false,         
    breaklines=true,                 
    captionpos=b,                    
    keepspaces=true,                 
    numbers=left,                    
    numbersep=5pt,                  
    showspaces=false,                
    showstringspaces=false,
    showtabs=false,                  
    tabsize=2
}
\lstset{style=mystyle,
        inputencoding=utf8,
        extendedchars=true,
        literate=%
        {é}{{\'{e}}}1}
\usepackage{mathrsfs}

\usepackage{titlesec}
\titleformat{\subsection}[hang]{\bfseries}{\thesubsection}{1em}{}
\titlespacing*{\subsection}{0pt}{3.5ex plus 1ex minus .2ex}{2.3ex plus .2ex}

\usepackage[utf8]{inputenc} % Required for inputting international characters
\usepackage[T1]{fontenc} % Use 8-bit encoding

\usepackage{geometry} % Required for adjusting page dimensions and margins
\geometry{
    paper=a4paper, % Paper size, change to letterpaper for US letter size
    top=2cm, % Top margin
    bottom=2.5cm, % Bottom margin
    left=2cm, % Left margin
    right=2cm, % Right margin
    headheight=0.75cm, % Header height
    footskip=1cm, % Space from the bottom margin to the baseline of the footer
    headsep=0.5cm, % Space from the top margin to the baseline of the header
    % showframe, % Uncomment to show how the type block is set on the page
}

\hypersetup{colorlinks=true, urlcolor=blue}

% \renewcommand{\thesection}{\Roman{section}}
\renewcommand{\thesubsection}{\alph{subsection})}
\newcommand{\ol}{\overline}
\newcommand{\nbsp}{\nobreakspace}

%------------------------------------------------------------------------------
%    TITLE SECTION
%------------------------------------------------------------------------------

\title{
    \vspace{10pt} % Whitespace
    \rule{\linewidth}{1pt}\\ % Thin top horizontal rule
    \vspace{10pt} % Whitespace
    {\huge DM2}\\ % The assignment title
    % \vspace{1pt} % Whitespace
    \rule{\linewidth}{1pt}\\ % Thick bottom horizontal rule
    \vspace{5pt} % Whitespace
}

\author{Élie Bellot des Minières} % Your name

\date{\normalsize{06 octobre 2024}} % Today's date (\today) or a custom date

\begin{document}
\maketitle % Print the title
\thispagestyle{empty}
\section{Questions}
\subsection*{Questions 1}
Pour $i$, l'indice de l'échantillon, sa valeur est $U(i\tau_{ech})$. Or, pour $U(t) = U_0\sin 2\pi ft$, $U(i\tau_{ech}) = U_0\sin (2\pi fi\tau_{ech})$.

\subsection*{Question 3}
\begin{lstlisting}
    52 49 46 46 2e 00 00 00 57 41 56 45 66 6d 74 20
    10 00 00 00 01 00 01 00 22 56 00 00 44 ac 00 00
    02 00 10 00 64 61 74 61 0a 00 00 00 d2 03 5e 06
    ff ff a2 f9 ff 7f 
\end{lstlisting}

\subsection*{Question 18}
On note $n$ le nombre de pistes et $l_1, \ldots, l_n$ le nombre d'échantillons total de chaque piste.
Tout d'abord, déterminons la complexité de \texttt{reduce\_track}.
Toutes les opérations hormis les boucles sont en $O(1)$.

Première boucle :

\begin{lstlisting}
    for (int i = 0; i < t->n_sounds; i++){
        s->n_samples += t->sounds[i]->n_samples;
    }
\end{lstlisting}

Deuxième boucle :

\begin{lstlisting}
    for (int i = 0; i < t->n_sounds; i++){
        for (int j = 0; j < t->sounds[i]->n_samples; j++){
            samples[n] = t->sounds[i]->samples[j];
            n++;
        }
    }
\end{lstlisting}

On observe que la première boucle est un grand $O$ de la deuxième puisque cette dernière possède une sous-boucle que la première n'a pas. Or, la deuxième boucle parcourt tous les échantillons de la piste. Elle est donc en $O(l_k)$.
La complexité de \texttt{reduce\_track} est donc aussi $O(l_k)$.

Pour ce qui est de \texttt{reduce\_mix} :

Première boucle :

\begin{lstlisting}
    for (int i = 0; i < m->n_track; i++){
        n = 0;
        for (int j = 0; j < m->tracks[i]->n_sounds; j++) {
            n += m->tracks[i]->sounds[j]->n_samples;
        }
        if (n > max_samples){
            max_samples = n;
        }
    }
\end{lstlisting}

Elle parcourt tous les sons de chaque piste.

\begin{lstlisting}
    int* samples_sum = calloc(max_samples, sizeof(int));
\end{lstlisting}

Est un grand $O$ de \texttt{max\_samples} qui est évidemment un grand $O$ de la première boucle.

Deuxième boucle :

\begin{lstlisting}
    for (int i = 0; i < m->n_track; i++){
        s_temp = reduce_track(m->tracks[i]);
        for (int j = 0; j < s_temp->n_samples; j++){
            samples_sum[j] += s_temp->samples[j] * m->vols[i];
        }
        free_sound(s_temp);
    }
\end{lstlisting}

On rappelle que \texttt{reduce\_track} est un grand $O$ de $l_k$. Et, la sous-boucle parcourt tous les échantillons de la piste. Elle est donc aussi un grand $O$ de $l_k$. La boucle extérieure parcourt toutes les $n$ pistes. Donc l'ensemble est de complexité : $O(l_1 + \ldots + l_n)$. On note que la première boucle de la fonction parcourt moins d'éléments que la deuxième. Donc, est aussi un grand $O$ de $l_1 + \ldots + l_n$. Pour finir, la dernière boucle est un grand $O$ de \texttt{max\_samples}, donc aussi de $l_1 + \ldots + l_n$. Au final, \texttt{reduce\_mix} est de complexité : $O(l_1 + \ldots + l_n)$.
\end{document}